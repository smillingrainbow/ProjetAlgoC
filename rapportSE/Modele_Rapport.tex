\documentclass[twoside,UTF8]{EPURapport}
\usepackage{listings}

%\renewcommand{\lstlistlistingname}{Liste des codes}
%\renewcommand{\lstlistingname}{Code}

%\addextratables{%
%	\lstlistoflistings
%}

%\swapAuthorsAndSupervisors



\thedocument{Rapport de Projet}{Le titre long}{Titre court}

\grade{Département Informatique\\ 3\ieme{} année\\ 2012 - 2013}

\authors{%
	\category{Étudiants}{%
		\name{Shimeng ZHANG} \mail{shimeng.zhang@etu.univ-tours.fr}
		\name{Natacha MARLIO-MARETTE} \mail{natacha.marlio-marette@etu.univ-tours.fr}
	}
	\details{DI3 2012 - 2013}
}

\supervisors{%
	\category{Encadrants}{%
		\name{Mickael Rousseau} \mail{mickael.rousseau@univ-tours.fr}
	}
	\details{Université François-Rabelais, Tours}
}

\abstracts{Description en français}
{Mots clés français}
{Description en anglais}
{Mots clés en anglais}

\begin{document}

%%%%%%%%%%% INTRODUCTION %%%%%%%%%%%%%%%%%%%%%%
\chapter{Introduction}

%%%%%%%%%%%%%%%%% CHAPITRE WINDOWS PHONE %%%%%%%%%

\chapter{Windows Phone}
\section{Historique}

\section{Matériel ciblé}
	\subsection{Pré-requis matériel}
%quel type de telephone faut-il pour cet OS?
	\subsection{Différents constructeurs}
%constructeurs qui commenercialisent
	
\section{Outils de programmation}
%outils de programmation et langage unique pour pc, WP, xBox et le web

\section{Marketplace}

%%%%%%%%%%%%% CHAPITRE OUTILS DE DEVELOPPEMENT
	
\chapter{Outils de développement}
	\section{Acquisition des outils}
	
	\section{\`Emulateur Windows Phone}
	
	\section{Visual Studio}
		\subsection{Création de projet}
		
%description des différents modèles de projets disponibles
		\subsection{Configuration de base}
%description des fichiers générés lors de la création d'un projet		
		
	\section{Expression Blend}
	
	\section{Test sur le matériel}
%comment tester son application sur son telephone, declaration avec Zune


%%%%%%%%%%% CHAPITRE SILVERLIGHT

\chapter{Silverlight}

	\section{Définition}
%definitions et explications de ce qu'est le framework silverlight
	
	%\section{Différentes classes}
	
	\section{Contrôles de positionnement}
		\subsection{La grille}
		\subsection{Le stackpanel}
		\subsection{Le canvas}

	\section{Contrôles d'entrée utilisateur}
		\subsection{TextBox}
		\subsection{CheckBox}
		\subsection{RadioButton}
		
	\section{Autres contrôles}
		\subsection{Button}
		\subsection{ContentControl}
		\subsection{ScrollViewver}
		\subsection{Formes}
		
	\section{Manipulation des images}
	
	\section{Geolocalisation}
		\subsection{Geolocalisation du smartphone}
		
		\subsection{Bing Maps}
	\section{Orientation de l'application}
		
%%%%%%%%%%%% CHAPITRE 4 LECTEUR RSS %%%%%%%%%%%%%%%%%%%%%%

\chapter{Lecteur RSS}
	\section{Flux RSS et applications}
		\subsection{Flux RSS}
%defintion d'un flux rss
		
		\subsection{Applications existantes}
%présentation des applis déjà existante sur le markerplace

	\section{Lecteur RSS}
	
%Explication sur notre appli : comment ca fonctionne, comment la faire .... 

%%%%%%%%%%%%% CONCLUSION %%%%%%%%%%%%%%%%%%%%

\chapter{Conclusion}

\annexes

\end{document}

